%----------------------------------------------------------------------------------------
%	Conclusion
%----------------------------------------------------------------------------------------
\chapter{Conclusion} % Main chapter title
\label{Chapter7} % For referencing the chapter elsewhere, use \ref{Chapter7}

%----------------------------------------------------------------------------------------
%	1. Project summary
%----------------------------------------------------------------------------------------

\section{Project summary}

The problem this project was created to resolve was to support a local gym which offers boxing, kickboxing and yoga services which is outgrowing their current system of tracking customers, their payments and attendance for lessons. The proposed solution was to create an online software which allows entry of member data, tracking of purchases and attendances for lessons while providing useful metrics for member signups, attendance and payments over several time scales.

Overall, this project has been a great success with results that have been overwhelming. The solution has been a comprehensive software capable of scaling as they grow while remaining simple for receptionists, coaches and managers to interact with.

The system also provides valuable, useful statistics for the business managers to make more informed business decisions. The accurate financial data has given the managers enough information to know if moving into another larger location is viable to accommodate the influx of members.

The improved and unified system has saved a great deal of time for the receptionist. She is now able to relax more with reduced pressure to remember customer information and is also able to perform other gym related tasks to assist the coaches.

A large component of this project which has resulted in this success has been through project planning and in depth discovery sessions which revealed the system requirements early on making focus during development far easier.

Research and planning also had a large impact on the efficiency of development. Having searched alternative products available on the market and effective design patterns for system architecture provided useful design and feature designs while reducing potentially wasted time reiterating over a more complex feature.


%----------------------------------------------------------------------------------------
%	2. Personal reflection
%----------------------------------------------------------------------------------------

\section{Personal reflection}

Although the project has been a success, there have been a few points where the process and resulting implementation could have been managed and worked on better.

Regular commits to the GitHub repository to better show progress over time and allow more flexibility in reversing any bugs or mistakes made. In combination with this, better use of branches for new features would have also contributed to a more efficient development process.

Having a developer's perspective of the software, some features which the managers and coaches may require were missed in the planning stages resulting in a less flexible system. One major oversight was in updating the prices for lessons. This was partially remedied through a new route being added to manually change the prices however would still require a developer to perform the change.

Meetings and time tracking at the beginning of the project was clear and simple. As development started and more complex features were being added, maintaining regularity and balancing workloads for other modules resulted in a less consistent development cycle which had a more significant impact when time had to be taken away for personal reasons. This created undue pressure where the mitigation plan was stressed beyond the planned mitigation.

If I was to work on this project or another project of this scale, I would make a few changes to my process and methodology for development.

The first would be the use of a protected main branch for the GitHub repository. This would have remedied the issues concerning regular and atomic commits where all changes would be tracked in their own smaller branches. The review time would be greatly reduced making for a more efficient and speedy development with as good or higher quality results.

Testing during and after development took a lot longer than anticipated. Planning around this through more time alotted to this task would be beneficial combined with time spent creating more automated testing. This would pay dividends as time goes on for a project where small components can be guaranteed to work.

Finally, a greater focus on the core functional features would have saved much time and work. With the functionality complete, user experience could be focused on with greater attention resulting in a far superior and user focused approach.

The University of Brighton has provided a great number of valuable modules which have contributed to the development of such a useful application. 

Web development during the full three years of study has given a solid foundation to build upon with valuable insights into the industry, best practices and functional tutorials.

Design patterns and system architecture was a module missing from my past knowledge and experience working in the industry. This module provided a focus on how to structure a product rather than build it. This key shift in focus has provided an effective angle to work from making larger projects more manageable.

The second year group project was eye opening to the requirements and needs of a project from a management perspective. Managing time and resources including other members of a team is vital for the creation of a complex system. This emphasises the planning and research phase of development showing how valuable a blueprint is to work from.

Private and individual study has been a huge benefit for me as no lecture, lab or module can provide all the necessary information for working on a project or in the industry. Experience working in the industry through a scholarship has garnered vast experience when in a professional setting. Considerations for customers and other developers is something that is not highly regarded in an academic setting.

Building personal project for both myself and private customers allows the implementation of the skills learned and provide a useful metric to see what knowledge you are lacking and an outlet to search for the solution.

%----------------------------------------------------------------------------------------
%	3. Future work
%----------------------------------------------------------------------------------------

\section{Future work}

The future of this project is bright with many features and improvements planned for the future. The current system is a great foundation to build upon with many features already planned and in development. The following is a list of features which are planned for the future.

\begin{itemize}
    \item Item purchases
    \item Basket system
    \item Hardware access for replacing till
\end{itemize}

The item purchases feature will allow the gym to track the sales of items such as gloves, wraps and other equipment. This will be a simple feature which will allow the receptionist to add items to a basket and checkout the customer with the selected items.

By allowing access to hardware such as till receipt printers, bar code scanners and cash draw will allow the gym to replace their current till system with the new software. This will allow the gym to have a single system for all their needs further unifying the systems and providing greater metrics.

Other concerns may become a concern such as multi-site and multi-gym support. This would require a more complex system to be developed to allow for the management of multiple sites and gyms. This would be a large undertaking and would require a great deal of planning and research to ensure the system is flexible enough to support the needs of the gym. These are all considerations which can be made in the future as the gym grows and expands in the years to come.
